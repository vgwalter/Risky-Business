% Options for packages loaded elsewhere
\PassOptionsToPackage{unicode}{hyperref}
\PassOptionsToPackage{hyphens}{url}
%
\documentclass[
]{article}
\usepackage{lmodern}
\usepackage{amssymb,amsmath}
\usepackage{ifxetex,ifluatex}
\ifnum 0\ifxetex 1\fi\ifluatex 1\fi=0 % if pdftex
  \usepackage[T1]{fontenc}
  \usepackage[utf8]{inputenc}
  \usepackage{textcomp} % provide euro and other symbols
\else % if luatex or xetex
  \usepackage{unicode-math}
  \defaultfontfeatures{Scale=MatchLowercase}
  \defaultfontfeatures[\rmfamily]{Ligatures=TeX,Scale=1}
\fi
% Use upquote if available, for straight quotes in verbatim environments
\IfFileExists{upquote.sty}{\usepackage{upquote}}{}
\IfFileExists{microtype.sty}{% use microtype if available
  \usepackage[]{microtype}
  \UseMicrotypeSet[protrusion]{basicmath} % disable protrusion for tt fonts
}{}
\makeatletter
\@ifundefined{KOMAClassName}{% if non-KOMA class
  \IfFileExists{parskip.sty}{%
    \usepackage{parskip}
  }{% else
    \setlength{\parindent}{0pt}
    \setlength{\parskip}{6pt plus 2pt minus 1pt}}
}{% if KOMA class
  \KOMAoptions{parskip=half}}
\makeatother
\usepackage{xcolor}
\IfFileExists{xurl.sty}{\usepackage{xurl}}{} % add URL line breaks if available
\IfFileExists{bookmark.sty}{\usepackage{bookmark}}{\usepackage{hyperref}}
\hypersetup{
  pdftitle={Sex and Financial Risk Aversion},
  pdfauthor={Victoria G. Walter},
  hidelinks,
  pdfcreator={LaTeX via pandoc}}
\urlstyle{same} % disable monospaced font for URLs
\usepackage[margin=1in]{geometry}
\usepackage{color}
\usepackage{fancyvrb}
\newcommand{\VerbBar}{|}
\newcommand{\VERB}{\Verb[commandchars=\\\{\}]}
\DefineVerbatimEnvironment{Highlighting}{Verbatim}{commandchars=\\\{\}}
% Add ',fontsize=\small' for more characters per line
\usepackage{framed}
\definecolor{shadecolor}{RGB}{248,248,248}
\newenvironment{Shaded}{\begin{snugshade}}{\end{snugshade}}
\newcommand{\AlertTok}[1]{\textcolor[rgb]{0.94,0.16,0.16}{#1}}
\newcommand{\AnnotationTok}[1]{\textcolor[rgb]{0.56,0.35,0.01}{\textbf{\textit{#1}}}}
\newcommand{\AttributeTok}[1]{\textcolor[rgb]{0.77,0.63,0.00}{#1}}
\newcommand{\BaseNTok}[1]{\textcolor[rgb]{0.00,0.00,0.81}{#1}}
\newcommand{\BuiltInTok}[1]{#1}
\newcommand{\CharTok}[1]{\textcolor[rgb]{0.31,0.60,0.02}{#1}}
\newcommand{\CommentTok}[1]{\textcolor[rgb]{0.56,0.35,0.01}{\textit{#1}}}
\newcommand{\CommentVarTok}[1]{\textcolor[rgb]{0.56,0.35,0.01}{\textbf{\textit{#1}}}}
\newcommand{\ConstantTok}[1]{\textcolor[rgb]{0.00,0.00,0.00}{#1}}
\newcommand{\ControlFlowTok}[1]{\textcolor[rgb]{0.13,0.29,0.53}{\textbf{#1}}}
\newcommand{\DataTypeTok}[1]{\textcolor[rgb]{0.13,0.29,0.53}{#1}}
\newcommand{\DecValTok}[1]{\textcolor[rgb]{0.00,0.00,0.81}{#1}}
\newcommand{\DocumentationTok}[1]{\textcolor[rgb]{0.56,0.35,0.01}{\textbf{\textit{#1}}}}
\newcommand{\ErrorTok}[1]{\textcolor[rgb]{0.64,0.00,0.00}{\textbf{#1}}}
\newcommand{\ExtensionTok}[1]{#1}
\newcommand{\FloatTok}[1]{\textcolor[rgb]{0.00,0.00,0.81}{#1}}
\newcommand{\FunctionTok}[1]{\textcolor[rgb]{0.00,0.00,0.00}{#1}}
\newcommand{\ImportTok}[1]{#1}
\newcommand{\InformationTok}[1]{\textcolor[rgb]{0.56,0.35,0.01}{\textbf{\textit{#1}}}}
\newcommand{\KeywordTok}[1]{\textcolor[rgb]{0.13,0.29,0.53}{\textbf{#1}}}
\newcommand{\NormalTok}[1]{#1}
\newcommand{\OperatorTok}[1]{\textcolor[rgb]{0.81,0.36,0.00}{\textbf{#1}}}
\newcommand{\OtherTok}[1]{\textcolor[rgb]{0.56,0.35,0.01}{#1}}
\newcommand{\PreprocessorTok}[1]{\textcolor[rgb]{0.56,0.35,0.01}{\textit{#1}}}
\newcommand{\RegionMarkerTok}[1]{#1}
\newcommand{\SpecialCharTok}[1]{\textcolor[rgb]{0.00,0.00,0.00}{#1}}
\newcommand{\SpecialStringTok}[1]{\textcolor[rgb]{0.31,0.60,0.02}{#1}}
\newcommand{\StringTok}[1]{\textcolor[rgb]{0.31,0.60,0.02}{#1}}
\newcommand{\VariableTok}[1]{\textcolor[rgb]{0.00,0.00,0.00}{#1}}
\newcommand{\VerbatimStringTok}[1]{\textcolor[rgb]{0.31,0.60,0.02}{#1}}
\newcommand{\WarningTok}[1]{\textcolor[rgb]{0.56,0.35,0.01}{\textbf{\textit{#1}}}}
\usepackage{graphicx,grffile}
\makeatletter
\def\maxwidth{\ifdim\Gin@nat@width>\linewidth\linewidth\else\Gin@nat@width\fi}
\def\maxheight{\ifdim\Gin@nat@height>\textheight\textheight\else\Gin@nat@height\fi}
\makeatother
% Scale images if necessary, so that they will not overflow the page
% margins by default, and it is still possible to overwrite the defaults
% using explicit options in \includegraphics[width, height, ...]{}
\setkeys{Gin}{width=\maxwidth,height=\maxheight,keepaspectratio}
% Set default figure placement to htbp
\makeatletter
\def\fps@figure{htbp}
\makeatother
\setlength{\emergencystretch}{3em} % prevent overfull lines
\providecommand{\tightlist}{%
  \setlength{\itemsep}{0pt}\setlength{\parskip}{0pt}}
\setcounter{secnumdepth}{-\maxdimen} % remove section numbering

\title{Sex and Financial Risk Aversion}
\author{Victoria G. Walter}
\date{}

\begin{document}
\maketitle

{
\setcounter{tocdepth}{3}
\tableofcontents
}
\hypertarget{introduction}{%
\section{Introduction}\label{introduction}}

\textbf{``How More Women on Wall Street Could Have Prevented the
Financial Crisis''}

This 2016 Fortune Magazine title and concept are obviously very
intriguing-- and controversial-- but is there any merit to the idea? Or
is it just click bait?

The article, an except of Newton-Small's new book ``Broad Influence: How
Women Are Changing the Way America Works,'' discusses the Great
Recession, it's causes, and the female regulators who were sent in to
fix the situation.

\emph{``The heart of the problem was risk, something mostly male Wall
Street seemed to take irresponsibly and something the female regulators
were sent in to mitigate. It's a cliched image: the straying reckless
man and a woman at home holding things together. But there is some
underlying truth in it. Neuroscience has shown links between risk taking
and testosterone, which is 15 times as prevalent in men as in women.
Many world leaders, from Bair to International Monetary Fund chief
Christine Lagarde to British Labour deput leader Harriet Harman, who was
then Prime Minister Gordon Brown's No.~2, go Japanese prime minister
Shinzo Abe, became convinced that if more women had been working in
senior Wall Street positions, the global financial crisis probably
wouldn't have happened. And many saw the crisis as a wake-up call for
Wall Street to diversify its ranks.''}

The article goes on to explain a number of studies supporting the idea
that women are more risk averse and that a greater female presence on
Wall Street changes the overall behavior of trading floors to make fewer
financially risky decisions.

Intrigued, I was inspired to create this report.

**I am curious regarding three points:

\begin{enumerate}
\def\labelenumi{\arabic{enumi}.}
\tightlist
\item
  Are women more risk averse investors?
\item
  Are women more likely to invest in risk averse assets and less likely
  to invest in riskier assets?
\item
  Do women invest more of their portfolios in risk averse assets and
  less in riskier assets?**
\end{enumerate}

Thus, I analyze investors' sex, self-reported level of risk aversion,
what assets they own, the market value of those assets, and how much of
their portfolios are invested in those asset.

In this report, you will find my methodology for this study, as well as
several plots and Cochran-Mantel-Haenszel tests to answer my questions.

\hypertarget{methodology}{%
\section{Methodology}\label{methodology}}

To do this, I select four common asset classes because of their distinct
levels of risk.

\hypertarget{explanation-of-asset-classes-and-risk}{%
\subsection{Explanation of Asset Classes and
Risk}\label{explanation-of-asset-classes-and-risk}}

An important note: Stocks (equities) are generally more risky than bonds
(debt instruments) because bondholders (those who own the company's
debt) are paid before shareholders (those who own equity, stocks) if the
company goes bankrupt. Thus, bondholders are more likely to receive
their money back on their investment than shareholders/stock owners, and
make bonds, generally, a less risky asset.

Stocks - individual shares of publicly traded equities. Of the four
asset classes, they generally the most volatile and they are the most
difficult to diversify and adjust for risk.

Stock Funds - a fund made up of stocks, including domestic stock funds,
growth funds, index funds, global stock funds, sector funds. Still
risky, but not as risky as stocks, because investors can diversify with
many different shares to adjust for risk in companies and sectors.

Combination Funds - funds that hold both stocks and bonds. Less risky,
because bonds are generally less risky than stocks, so the combination
fund is reasonably diversified and risk adjusted.

Treasury Bond Funds - Treasury bonds are bonds backed by the United
States government and are safest asset to own (except cash) because the
United States has never defaulted. A treasury bond fund is just a fund
made up of these bonds, most likely with different rates to maturity,
such as a 2-year bond, 10-year bond, and 30-year bond.

\hypertarget{data}{%
\subsection{Data}\label{data}}

I use the most recent The Survey of Consumer Finances (2016), a survey
to assess the economic and financial health of those in the United
States, including families' balance sheets, pensions, incomes, and
demographic information. It is conducted by the Federal Reserve Bank of
New York every four years or so, partially in person and partially over
the phone. The data set is very useful because it has a large sample
size -- over 30,000 respondents -- and it offers a great depth of
information -- over 6,000 variables.

However, it is important to remember that the survey is voluntary, and
very long and invasive, so population skews very high towards those that
are older, likely retired, and likely to have more time on their hands.
Additionally, the survey is predominantly White and thus isn't properly
representative of the United States racial makeup in 2016 -- also a flaw
of it being a very long voluntary survey.

\hypertarget{cleaning}{%
\section{Cleaning}\label{cleaning}}

First I import and clean my data -- the 2016 Survey of Consumer
Finances. \#\# Importing 2016 Survey of Consumer Finances Data

\begin{Shaded}
\begin{Highlighting}[]
\NormalTok{scf_data <-}\StringTok{ }\KeywordTok{read_feather}\NormalTok{(}\StringTok{'../data/scf_data.ftr'}\NormalTok{)}
\end{Highlighting}
\end{Shaded}

\hypertarget{select-variables}{%
\subsection{Select Variables}\label{select-variables}}

Once the data is imported, I select the specific columns I am Using for
my analysis and rename them so they make sense. I chose the following
variables:

Variable Description

Sex\ldots\ldots\ldots\ldots\ldots\ldots\ldots\ldots\ldots\ldots\ldots\ldots\ldots\ldots\ldots\ldots\ldots\ldots\ldots What
is the sex of the individual who makes the financial decisions in the
Respondent's family?

Financial\_Risk\_Willingness\ldots\ldots\ldots\ldots\ldots..How willing
to take financial risks is the individual who makes the financial
decisions in the Respondent's family? On a Scale of -1 (Not at All
Willing) to 10 (Very Willing)

Owns\_Stocks\ldots\ldots\ldots\ldots\ldots\ldots\ldots\ldots\ldots\ldots..Does
the respondent own stocks?

Owns\_Stock\_Funds\ldots\ldots\ldots\ldots\ldots\ldots\ldots\ldots\ldots Does
the respondent own stock funds?

Owns\_Combination\_Funds Does the respondent own combination funds?

Owns\_Treasury\_Bonds\_Funds Does the respondent own treasury bond
funds?

TMV\_Stocks Total Market Value of the Stocks the Respondent Owns in 2016
U.S. Dollars

TMV\_Stock\_Funds Total Market Value of the Stock Funds the Respondent
Owns in 2016 U.S. Dollars

TMV\_Combination\_Funds Total Market Value of the Combination Funds the
Respondent Owns in 2016 U.S. Dollars

TMV\_Treasury\_Bond\_Funds Total Market Value of the Treasury Bond Funds
Funds the Respondent Owns in 2016 U.S. Dollars

\begin{Shaded}
\begin{Highlighting}[]
\NormalTok{scf_data }\OperatorTok
\StringTok{  }\KeywordTok{select}\NormalTok{(}\StringTok{"Sex"}\NormalTok{ =}\StringTok{ "X8000"}\NormalTok{,}
         \StringTok{"Financial_Risk_Willingness"}\NormalTok{ =}\StringTok{ "X7557"}\NormalTok{, }
         \StringTok{"Owns_Stocks"}\NormalTok{ =}\StringTok{ "X3913"}\NormalTok{, }
         \StringTok{"Owns_Stock_Funds"}\NormalTok{ =}\StringTok{ "X3821"}\NormalTok{, }
         \StringTok{"Owns_Combination_Funds"}\NormalTok{ =}\StringTok{ "X3829"}\NormalTok{, }
         \StringTok{"Owns_Treasury_Bond_Funds"}\NormalTok{ =}\StringTok{ "X3825"}\NormalTok{, }
         \StringTok{"TMV_Stocks"}\NormalTok{ =}\StringTok{ "X3915"}\NormalTok{, }
         \StringTok{"TMV_Stock_Funds"}\NormalTok{ =}\StringTok{ "X3822"}\NormalTok{, }
         \StringTok{"TMV_Combination_Funds"}\NormalTok{ =}\StringTok{ "X3830"}\NormalTok{,}
         \StringTok{"TMV_Treasury_Bond_Funds"}\NormalTok{ =}\StringTok{ "X3826"}\NormalTok{,)}
\end{Highlighting}
\end{Shaded}

\hypertarget{recode-and-factor-existing-categorical-variables}{%
\subsection{Recode and Factor Existing Categorical
Variables}\label{recode-and-factor-existing-categorical-variables}}

Next, I recode the variables that are categorical so that they make
sense, and make them factored

\begin{Shaded}
\begin{Highlighting}[]
\NormalTok{scf_data}\OperatorTok{$}\NormalTok{Sex =}\StringTok{ }\NormalTok{dplyr}\OperatorTok{::}\KeywordTok{recode_factor}\NormalTok{(}\KeywordTok{as.character}\NormalTok{(scf_data}\OperatorTok{$}\NormalTok{Sex), }\StringTok{"1"}\NormalTok{=}\StringTok{"Female"}\NormalTok{,}\StringTok{"5"}\NormalTok{=}\StringTok{"Male"}\NormalTok{)}

\NormalTok{scf_data}\OperatorTok{$}\NormalTok{Owns_Stocks <-}\StringTok{ }\NormalTok{dplyr}\OperatorTok{::}\KeywordTok{recode}\NormalTok{(}\KeywordTok{as.character}\NormalTok{(scf_data}\OperatorTok{$}\NormalTok{Owns_Stocks), }\StringTok{"1"}\NormalTok{=}\StringTok{"Owns"}\NormalTok{, }\StringTok{"5"}\NormalTok{=}\StringTok{"Does Not Own"}\NormalTok{)}
\end{Highlighting}
\end{Shaded}

For the fund variables -- stock funds, combination funds, and treasury
bond funds -- the coding is as follows: 1 = Owns \_\_\_\_\_ funds, 5 =
Doesn't own \_\_\_\_\_ funds, but owns other funds, 0 = Doesn't own any
funds

\begin{Shaded}
\begin{Highlighting}[]
\NormalTok{scf_data}\OperatorTok{$}\NormalTok{Owns_Stock_Funds <-}\StringTok{ }\NormalTok{dplyr}\OperatorTok{::}\KeywordTok{recode}\NormalTok{(}\KeywordTok{as.character}\NormalTok{(scf_data}\OperatorTok{$}\NormalTok{Owns_Stock_Funds),}\StringTok{"1"}\NormalTok{=}\StringTok{"Owns"}\NormalTok{, }
                                                                                \StringTok{"5"}\NormalTok{=}\StringTok{"Does Not Own"}\NormalTok{,}
                                                                                \StringTok{"0"}\NormalTok{=}\StringTok{"Does Not Own"}\NormalTok{)}

\NormalTok{scf_data}\OperatorTok{$}\NormalTok{Owns_Combination_Funds <-}\StringTok{ }\NormalTok{dplyr}\OperatorTok{::}\KeywordTok{recode}\NormalTok{(}\KeywordTok{as.character}\NormalTok{(scf_data}\OperatorTok{$}\NormalTok{Owns_Combination_Funds), }\StringTok{"1"}\NormalTok{=}\StringTok{"Owns"}\NormalTok{, }
                                                                                    \StringTok{"5"}\NormalTok{=}\StringTok{"Does Not Own"}\NormalTok{,}
                                                                                    \StringTok{"0"}\NormalTok{=}\StringTok{"Does Not Own"}\NormalTok{)}

\NormalTok{scf_data}\OperatorTok{$}\NormalTok{Owns_Treasury_Bond_Funds <-}\StringTok{ }\NormalTok{dplyr}\OperatorTok{::}\KeywordTok{recode}\NormalTok{(}\KeywordTok{as.character}\NormalTok{(scf_data}\OperatorTok{$}\NormalTok{Owns_Treasury_Bond_Funds), }
                                                                                    \StringTok{"1"}\NormalTok{=}\StringTok{"Owns"}\NormalTok{, }
                                                                                    \StringTok{"5"}\NormalTok{=}\StringTok{"Does Not Own"}\NormalTok{,}
                                                                                    \StringTok{"0"}\NormalTok{=}\StringTok{"Does Not Own"}\NormalTok{)}
\end{Highlighting}
\end{Shaded}

\hypertarget{create-new-variables}{%
\subsection{Create New Variables}\label{create-new-variables}}

Additionally, I am creating a few new variables to account for how much
a Respondent owns in one asset relative to the other assets.

First, I create ``TMV\_Investments,'' or the Total Market Value of
Stocks, Stock Funds, Combination Funds, and Treasury Bond Funds the
Respondent owns added together.

Then, I calculate the percentage of the Respondent's portfolio in each
asset. This is done by dividing the Total Market Value of the Asset by
the Total Market Value of all four assets combined and then multiplied
by 100\%.

\begin{verbatim}
     Variable                                               Description
"Percent_Treasury_Bonds"           Percentage of the Respondent's Portfolio in Treasury Bond Funds
"Percent_Combination_Funds"        Percentage of the Respondent's Portfolio in Combination Funds
"Percent_Stock_Funds"              Percentage of the Respondent's Portfolio in Stock Funds
"Percent_Stocks"                   Percentage of the Respondent's Portfolio in Stocks
\end{verbatim}

\begin{Shaded}
\begin{Highlighting}[]
\NormalTok{scf_data }\OperatorTok
\StringTok{  }\KeywordTok{mutate}\NormalTok{(}\DataTypeTok{TMV_Investments =}\NormalTok{ TMV_Treasury_Bond_Funds }\OperatorTok{+}\StringTok{ }\NormalTok{TMV_Stocks }\OperatorTok{+}\StringTok{ }\NormalTok{TMV_Stock_Funds }\OperatorTok{+}\StringTok{ }\NormalTok{TMV_Combination_Funds,}
         \DataTypeTok{Percent_Treasury_Bond_Funds =}\NormalTok{ TMV_Treasury_Bond_Funds }\OperatorTok{/}\StringTok{ }\NormalTok{TMV_Investments }\OperatorTok{*}\StringTok{ }\DecValTok{100}\NormalTok{,}
         \DataTypeTok{Percent_Combination_Funds =}\NormalTok{ TMV_Combination_Funds }\OperatorTok{/}\StringTok{ }\NormalTok{TMV_Investments }\OperatorTok{*}\StringTok{ }\DecValTok{100}\NormalTok{,}
         \DataTypeTok{Percent_Stock_Funds =}\NormalTok{ TMV_Stock_Funds }\OperatorTok{/}\StringTok{ }\NormalTok{TMV_Investments }\OperatorTok{*}\StringTok{ }\DecValTok{100}\NormalTok{,}
         \DataTypeTok{Percent_Stocks =}\NormalTok{ TMV_Stocks }\OperatorTok{/}\StringTok{ }\NormalTok{TMV_Investments }\OperatorTok{*}\StringTok{ }\DecValTok{100}\NormalTok{)}
\end{Highlighting}
\end{Shaded}

Next, I create the ``Risk\_Aversion'' variable, which takes the
Financial Risk Willingness scale of -1 to 10 and groups it into a
condensed categorical variable.

\begin{verbatim}
  Risk Aversion (New Variable)              Financial Risk Willingness (Variable from SCF)
     Risk Averse                                         -1, 1, 2, 3
     Risk Neutral                                            4, 5, 6
     Risk Tolerant                                       7, 8, 9, 10
\end{verbatim}

\begin{Shaded}
\begin{Highlighting}[]
\NormalTok{scf_data}\OperatorTok{$}\NormalTok{Risk_Aversion <-}\StringTok{ }\KeywordTok{cut}\NormalTok{(scf_data}\OperatorTok{$}\NormalTok{Financial_Risk_Willingness,}
                                                   \DataTypeTok{breaks =} \KeywordTok{c}\NormalTok{(}\OperatorTok{-}\OtherTok{Inf}\NormalTok{, }\DecValTok{4}\NormalTok{, }\DecValTok{7}\NormalTok{, }\OtherTok{Inf}\NormalTok{),}
                                                   \DataTypeTok{labels =} \KeywordTok{c}\NormalTok{(}\StringTok{"Risk Averse"}\NormalTok{, }\StringTok{"Risk Neutral"}\NormalTok{, }\StringTok{"Risk Tolerant"}\NormalTok{))}
\end{Highlighting}
\end{Shaded}

Then, I order the Risk Aversion variable from risk averse to risk
tolerant.

\begin{Shaded}
\begin{Highlighting}[]
\NormalTok{scf_data}\OperatorTok{$}\NormalTok{Risk_Aversion <-}\StringTok{ }\KeywordTok{ordered}\NormalTok{(scf_data}\OperatorTok{$}\NormalTok{Risk_Aversion, }\DataTypeTok{levels =} \KeywordTok{c}\NormalTok{(}\StringTok{"Risk Averse"}\NormalTok{, }\StringTok{"Risk Neutral"}\NormalTok{, }\StringTok{"Risk Tolerant"}\NormalTok{))}
\end{Highlighting}
\end{Shaded}

\hypertarget{filtering-survey-data-for-only-respondents-that-own-investments}{%
\subsection{Filtering Survey Data for Only Respondents that Own
Investments}\label{filtering-survey-data-for-only-respondents-that-own-investments}}

The final step in cleaning is filtering only for Respondents who's Total
Market Value of Investments (Stocks + Stock Funds +Combination Funds +
Treasury Bond Funds) is Greater than 0. Therefore this analysis will
only be run on Respondents that own at least 1 of the four assets.

\begin{Shaded}
\begin{Highlighting}[]
\NormalTok{scf_data_cleaned <-}\StringTok{ }\NormalTok{scf_data }\OperatorTok
\StringTok{  }\KeywordTok{filter}\NormalTok{(TMV_Investments}\OperatorTok{>}\DecValTok{0}\NormalTok{)}
\end{Highlighting}
\end{Shaded}

This significantly reduces the data set, from 31,240 observations to
9,156 From this point on in the analysis, I will be referring to these
filtered Respondents as ``Investors.''

\hypertarget{visualizations}{%
\section{Visualizations}\label{visualizations}}

\hypertarget{the-number-of-male-and-female-investors-that-own-each-asset}{%
\subsection{The Number of Male and Female Investors That Own Each
Asset}\label{the-number-of-male-and-female-investors-that-own-each-asset}}

\hypertarget{table-1-count_asset}{%
\subsubsection{Table 1: count\_asset}\label{table-1-count_asset}}

I start my analysis by creating a pivot table, ``count\_assets,'' of the
number of male and female investors that own and do not each asset.

\begin{Shaded}
\begin{Highlighting}[]
\NormalTok{count_asset <-}\StringTok{ }\NormalTok{scf_data_cleaned }\OperatorTok
\StringTok{  }\KeywordTok{select}\NormalTok{(}\DataTypeTok{Stocks=}\NormalTok{Owns_Stocks, }\DataTypeTok{Stock_Funds=}\NormalTok{Owns_Stock_Funds, }\DataTypeTok{Combination_Funds=}\NormalTok{Owns_Combination_Funds, }\DataTypeTok{Treasury_Bond_Funds=}\NormalTok{Owns_Treasury_Bond_Funds, Sex) }\OperatorTok
\StringTok{  }\KeywordTok{pivot_longer}\NormalTok{(}\OperatorTok{-}\NormalTok{Sex, }\DataTypeTok{names_to=}\StringTok{'Asset'}\NormalTok{, }\DataTypeTok{values_to=}\StringTok{'Response'}\NormalTok{) }\OperatorTok
\StringTok{  }\KeywordTok{count}\NormalTok{(Sex, Asset, Response)}
\end{Highlighting}
\end{Shaded}

I factor and order the levels of the Asset variable in ``count\_assets''
so that the assets are in order of riskiest to least risky.

\begin{Shaded}
\begin{Highlighting}[]
\NormalTok{count_asset}\OperatorTok{$}\NormalTok{Asset <-}\StringTok{ }\KeywordTok{factor}\NormalTok{(count_asset}\OperatorTok{$}\NormalTok{Asset, }\DataTypeTok{levels =} \KeywordTok{c}\NormalTok{(}\StringTok{"Stocks"}\NormalTok{, }\StringTok{"Stock_Funds"}\NormalTok{, }\StringTok{"Combination_Funds"}\NormalTok{, }\StringTok{"Treasury_Bond_Funds"}\NormalTok{, }\DataTypeTok{ordered =} \OtherTok{TRUE}\NormalTok{))}
\end{Highlighting}
\end{Shaded}

\hypertarget{graph-1}{%
\subsubsection{Graph 1}\label{graph-1}}

Then I make a bar plot with ``count\_assets'' using ggplot2 to visualize
the Number of Investors who own each asset, broken down by sex. I use
the Wes Anderson color palette Grand Budapest 1 from one of my favorite
movies.

\begin{Shaded}
\begin{Highlighting}[]
\KeywordTok{ggplot}\NormalTok{(count_asset, }\KeywordTok{aes}\NormalTok{(}\DataTypeTok{x=}\NormalTok{Sex, }\DataTypeTok{y=}\NormalTok{n, }\DataTypeTok{fill=}\NormalTok{Response))}\OperatorTok{+}
\StringTok{  }\KeywordTok{geom_bar}\NormalTok{(}\DataTypeTok{stat =} \StringTok{'identity'}\NormalTok{, }\DataTypeTok{position =} \StringTok{'dodge'}\NormalTok{)}\OperatorTok{+}
\StringTok{  }\KeywordTok{facet_grid}\NormalTok{(}\OperatorTok{~}\NormalTok{Asset)}\OperatorTok{+}
\StringTok{  }\KeywordTok{labs}\NormalTok{(}\DataTypeTok{title =} \StringTok{"Graph 1: The Number of Male and Female Investors that Own Each Asset, }\CharTok{\textbackslash{}n}\StringTok{   from Riskiest to Least Risky Asset (left to right)"}\NormalTok{, }\DataTypeTok{y =} \StringTok{"Number of Investors"}\NormalTok{, }\DataTypeTok{x =} \StringTok{"Asset"}\NormalTok{) }\OperatorTok{+}
\StringTok{  }\KeywordTok{geom_text}\NormalTok{(}\KeywordTok{aes}\NormalTok{(}\DataTypeTok{label=}\NormalTok{n), }\DataTypeTok{position=}\KeywordTok{position_dodge}\NormalTok{(}\DataTypeTok{width=}\FloatTok{0.9}\NormalTok{), }\DataTypeTok{vjust=}\OperatorTok{-}\FloatTok{0.25}\NormalTok{)}\OperatorTok{+}
\StringTok{  }\KeywordTok{theme_classic}\NormalTok{()}\OperatorTok{+}
\StringTok{  }\KeywordTok{theme}\NormalTok{(}\DataTypeTok{axis.text.x =} \KeywordTok{element_text}\NormalTok{(}\DataTypeTok{angle =} \DecValTok{20}\NormalTok{, }\DataTypeTok{vjust =} \FloatTok{1.2}\NormalTok{, }\DataTypeTok{hjust=}\DecValTok{1}\NormalTok{)) }\OperatorTok{+}
\StringTok{  }\KeywordTok{theme}\NormalTok{(}\DataTypeTok{legend.position =} \StringTok{'top'}\NormalTok{)}\OperatorTok{+}
\StringTok{  }\KeywordTok{theme}\NormalTok{(}\DataTypeTok{plot.title =} \KeywordTok{element_text}\NormalTok{(}\DataTypeTok{hjust =} \FloatTok{0.4}\NormalTok{))}\OperatorTok{+}
\StringTok{  }\KeywordTok{scale_fill_manual}\NormalTok{(}\DataTypeTok{values =} \KeywordTok{wes_palette}\NormalTok{(}\StringTok{"GrandBudapest1"}\NormalTok{, }\DataTypeTok{n =} \DecValTok{3}\NormalTok{))}
\end{Highlighting}
\end{Shaded}

\includegraphics{scf_analysis_files/figure-latex/unnamed-chunk-11-1.pdf}

\hypertarget{insights-from-graph-1}{%
\subsubsection{Insights from Graph 1:}\label{insights-from-graph-1}}

There are far more male investors than female investors in this data
set. There are more men that own each asset than women, and therefore
the percentage of men that own the asset is greater than the percentage
of women that own the asset, for all four assets. Stocks are the most
popularly owned asset, followed by Stock Funds, Treasury Bond Funds, and
Combination Funds.

1,105 women own stocks and 512 do not, compared to 5,612 men who own
stocks and 1,927 who do not. 841 women own stock funds and 776 do not,
versus 4,247 men who own stock funds and 3,292 who do not. 102 women own
combination funds and 1,515 do not, compared to 623 men who own
combination funds and 6,916 who do not. 98 women own treasury bond funds
and 1,519 do not, versus 679 men who own treasury bond funds and 6,860
who do not.

\hypertarget{the-percentage-of-male-and-female-investors-by-risk-aversion}{%
\subsection{The Percentage of Male and Female Investors by Risk
Aversion}\label{the-percentage-of-male-and-female-investors-by-risk-aversion}}

\hypertarget{table-2-percent_sex_risk}{%
\subsubsection{Table 2:
percent\_sex\_risk}\label{table-2-percent_sex_risk}}

Next, I am interested in how male and female investors describe their
level of risk aversion. To do this, I created a pivot table,
``percent\_sex\_risk,'' and then created a ``Percentage'' column. It
makes more sense to use percentages instead of number of investors for
this analysis, because there are so many more male investors than female
investors.

\begin{Shaded}
\begin{Highlighting}[]
\NormalTok{percent_sex_risk <-}\StringTok{ }\NormalTok{scf_data_cleaned }\OperatorTok
\StringTok{  }\KeywordTok{select}\NormalTok{(Sex, Risk_Aversion) }\OperatorTok
\StringTok{  }\KeywordTok{group_by}\NormalTok{(Sex, Risk_Aversion) }\OperatorTok
\StringTok{  }\KeywordTok{summarise}\NormalTok{(}\DataTypeTok{Number_of_Responses=}\KeywordTok{n}\NormalTok{()) }\OperatorTok
\StringTok{  }\KeywordTok{group_by}\NormalTok{(Sex) }\OperatorTok
\StringTok{  }\KeywordTok{mutate}\NormalTok{(}\DataTypeTok{Percentage=}\NormalTok{Number_of_Responses}\OperatorTok{/}\KeywordTok{sum}\NormalTok{(Number_of_Responses)}\OperatorTok{*}\DecValTok{100}\NormalTok{)}
\end{Highlighting}
\end{Shaded}

\begin{verbatim}
## `summarise()` regrouping output by 'Sex' (override with `.groups` argument)
\end{verbatim}

\hypertarget{graph-2}{%
\subsubsection{Graph 2}\label{graph-2}}

I made a barplot with ``percent\_sex\_risk'' to visualize if there are
any noticeable differences between how the sexes report their level of
risk averison.

\begin{Shaded}
\begin{Highlighting}[]
\KeywordTok{ggplot}\NormalTok{(percent_sex_risk, }\KeywordTok{aes}\NormalTok{(}\DataTypeTok{x =}\NormalTok{ Risk_Aversion, }\DataTypeTok{y =}\NormalTok{ Percentage, }\DataTypeTok{fill =}\NormalTok{ Sex))}\OperatorTok{+}\StringTok{ }
\StringTok{  }\KeywordTok{geom_bar}\NormalTok{(}\DataTypeTok{stat=}\StringTok{'identity'}\NormalTok{, }\DataTypeTok{position=}\StringTok{'dodge'}\NormalTok{)}\OperatorTok{+}
\StringTok{  }\KeywordTok{labs}\NormalTok{(}\DataTypeTok{x =} \StringTok{"Risk Aversion"}\NormalTok{, }\DataTypeTok{y =} \StringTok{"Percentage"}\NormalTok{, }\DataTypeTok{title =} \StringTok{"Graph 2: Percentage of Male and Female Investors by Risk Aversion"}\NormalTok{)}\OperatorTok{+}\StringTok{   }
\StringTok{  }\KeywordTok{geom_text}\NormalTok{(}\KeywordTok{aes}\NormalTok{(}\DataTypeTok{label=}\KeywordTok{round}\NormalTok{(Percentage, }\DataTypeTok{digits =} \DecValTok{2}\NormalTok{)), }\DataTypeTok{position=}\KeywordTok{position_dodge}\NormalTok{(}\DataTypeTok{width=}\FloatTok{0.9}\NormalTok{), }\DataTypeTok{vjust=}\OperatorTok{-}\FloatTok{0.25}\NormalTok{)}\OperatorTok{+}
\StringTok{  }\KeywordTok{theme_classic}\NormalTok{()}\OperatorTok{+}
\StringTok{  }\KeywordTok{theme}\NormalTok{(}\DataTypeTok{legend.position =} \StringTok{"top"}\NormalTok{)}\OperatorTok{+}
\StringTok{  }\KeywordTok{theme}\NormalTok{(}\DataTypeTok{axis.text.x =} \KeywordTok{element_text}\NormalTok{(}\DataTypeTok{angle =} \DecValTok{15}\NormalTok{, }\DataTypeTok{vjust =} \FloatTok{1.2}\NormalTok{, }\DataTypeTok{hjust=}\DecValTok{1}\NormalTok{))}\OperatorTok{+}\StringTok{ }
\StringTok{  }\KeywordTok{scale_fill_manual}\NormalTok{(}\DataTypeTok{values =} \KeywordTok{wes_palette}\NormalTok{(}\StringTok{"GrandBudapest1"}\NormalTok{, }\DataTypeTok{n =} \DecValTok{2}\NormalTok{))}
\end{Highlighting}
\end{Shaded}

\includegraphics{scf_analysis_files/figure-latex/unnamed-chunk-13-1.pdf}

\hypertarget{insights-from-graph-2}{%
\subsubsection{Insights from Graph 2}\label{insights-from-graph-2}}

Male investors are risk tolerant at twice the level of female investors
-- 28.57\% vs.~14.41\%.

There is a much greater percentage of female investors who are risk
averse than men -- 31.23\% vs.~21.16\%.

A slightly greater percentage of female investors are risk neutral than
male investors, but it is close -- 54.36\% vs.~50.27\%.

\hypertarget{the-percentage-of-investors-that-own-each-asset-by-sex-and-risk-aversion}{%
\subsection{The Percentage of Investors that Own Each Asset, by Sex and
Risk
Aversion}\label{the-percentage-of-investors-that-own-each-asset-by-sex-and-risk-aversion}}

\hypertarget{table-3-percent_own}{%
\subsubsection{Table 3: percent\_own}\label{table-3-percent_own}}

Next, I create the pivot table, ``percent\_own,'' to observe what
percentage of investors who own each asset identity as different levels
of risk aversion. This is useful to see, considering that the each asset
has a different level of risk. Do those that own risky assets have a
higher percentage of people that are risk tolerant? Do those that own
less risky assets have a higher percentage of people that are risk
averse? ``Percent\_own'' also breaks this information down by sex.

\begin{Shaded}
\begin{Highlighting}[]
\NormalTok{percent_own <-}\StringTok{ }\NormalTok{scf_data_cleaned }\OperatorTok
\StringTok{  }\KeywordTok{select}\NormalTok{(}\DataTypeTok{Stocks=}\NormalTok{Owns_Stocks, }\DataTypeTok{Stock_Funds=}\NormalTok{Owns_Stock_Funds, }\DataTypeTok{Combination_Funds=}\NormalTok{Owns_Combination_Funds, }\DataTypeTok{Treasury_Bond_Funds=}\NormalTok{Owns_Treasury_Bond_Funds, Sex, Risk_Aversion) }\OperatorTok
\StringTok{  }\KeywordTok{pivot_longer}\NormalTok{(}\OperatorTok{-}\KeywordTok{c}\NormalTok{(Sex, Risk_Aversion), }\DataTypeTok{names_to=} \StringTok{'Asset'}\NormalTok{,  }\DataTypeTok{values_to=}\StringTok{'Response'}\NormalTok{) }\OperatorTok
\StringTok{  }\KeywordTok{count}\NormalTok{(Sex, Asset, Risk_Aversion, Response) }\OperatorTok
\StringTok{  }\KeywordTok{group_by}\NormalTok{(Sex, Asset, Risk_Aversion) }\OperatorTok
\StringTok{  }\KeywordTok{mutate}\NormalTok{(}\DataTypeTok{percent =}\NormalTok{ n }\OperatorTok{/}\StringTok{ }\KeywordTok{sum}\NormalTok{(n) }\OperatorTok{*}\StringTok{ }\DecValTok{100}\NormalTok{)}
\end{Highlighting}
\end{Shaded}

I make Asset column a factor and order it from the most risky asset to
least risky asset.

\begin{Shaded}
\begin{Highlighting}[]
\NormalTok{percent_own}\OperatorTok{$}\NormalTok{Asset <-}\StringTok{ }\KeywordTok{factor}\NormalTok{(percent_own}\OperatorTok{$}\NormalTok{Asset, }\DataTypeTok{levels =} \KeywordTok{c}\NormalTok{(}\StringTok{"Stocks"}\NormalTok{, }\StringTok{"Stock_Funds"}\NormalTok{, }\StringTok{"Combination_Funds"}\NormalTok{, }\StringTok{"Treasury_Bond_Funds"}\NormalTok{, }\DataTypeTok{ordered =} \OtherTok{TRUE}\NormalTok{))}
\end{Highlighting}
\end{Shaded}

I filter the Owns column in ``percent\_own,'' to create
``percent\_own\_filtered,'' so that only investors that own that
particular asset are in the data.

\begin{Shaded}
\begin{Highlighting}[]
\NormalTok{percent_own_filtered <-}\StringTok{ }\NormalTok{percent_own }\OperatorTok
\StringTok{  }\KeywordTok{filter}\NormalTok{(Response}\OperatorTok{==}\StringTok{"Owns"}\NormalTok{)}
\end{Highlighting}
\end{Shaded}

\hypertarget{graph-3}{%
\subsubsection{Graph 3}\label{graph-3}}

I made a barplot to visualize ``percent\_own\_filtered'' and see if the
investors in this data set invest according to their risk tolerance.

\begin{Shaded}
\begin{Highlighting}[]
\KeywordTok{ggplot}\NormalTok{(percent_own_filtered, }\KeywordTok{aes}\NormalTok{(}\DataTypeTok{x =}\NormalTok{ Risk_Aversion, }\DataTypeTok{y =}\NormalTok{ percent, }\DataTypeTok{fill =}\NormalTok{ Risk_Aversion))}\OperatorTok{+}
\StringTok{  }\KeywordTok{geom_bar}\NormalTok{(}\DataTypeTok{stat=}\StringTok{'identity'}\NormalTok{, }\DataTypeTok{position =} \StringTok{'dodge'}\NormalTok{)}\OperatorTok{+}
\StringTok{  }\KeywordTok{facet_wrap}\NormalTok{(}\OperatorTok{~}\NormalTok{Asset }\OperatorTok{+}\StringTok{ }\NormalTok{Sex, }\DataTypeTok{ncol=}\DecValTok{4}\NormalTok{)}\OperatorTok{+}
\StringTok{  }\KeywordTok{labs}\NormalTok{(}\DataTypeTok{x =} \StringTok{"Risk Aversion"}\NormalTok{, }\DataTypeTok{y =} \StringTok{"Percentage"}\NormalTok{, }\DataTypeTok{title =} \StringTok{"Graph 3: Percentage of Investors that Own Each Asset, }\CharTok{\textbackslash{}n}\StringTok{ by Risk Aversion and Sex"}\NormalTok{)}\OperatorTok{+}
\StringTok{  }\KeywordTok{geom_text}\NormalTok{(}\KeywordTok{aes}\NormalTok{(}\DataTypeTok{label=}\KeywordTok{round}\NormalTok{(percent, }\DataTypeTok{digits =} \DecValTok{2}\NormalTok{)), }\DataTypeTok{position=}\KeywordTok{position_dodge}\NormalTok{(}\DataTypeTok{width=}\FloatTok{0.9}\NormalTok{), }\DataTypeTok{vjust=}\OperatorTok{-}\FloatTok{0.25}\NormalTok{)}\OperatorTok{+}
\StringTok{  }\KeywordTok{theme_classic}\NormalTok{()}\OperatorTok{+}
\StringTok{  }\KeywordTok{theme}\NormalTok{(}\DataTypeTok{axis.text.x =} \KeywordTok{element_blank}\NormalTok{())}\OperatorTok{+}\StringTok{ }
\StringTok{  }\KeywordTok{theme}\NormalTok{(}\DataTypeTok{axis.ticks.x =} \KeywordTok{element_blank}\NormalTok{())}\OperatorTok{+}
\StringTok{  }\KeywordTok{theme}\NormalTok{(}\DataTypeTok{legend.position =} \StringTok{"top"}\NormalTok{)}\OperatorTok{+}
\StringTok{  }\KeywordTok{scale_fill_manual}\NormalTok{(}\DataTypeTok{values =} \KeywordTok{wes_palette}\NormalTok{(}\StringTok{"GrandBudapest1"}\NormalTok{, }\DataTypeTok{n =} \DecValTok{3}\NormalTok{))}
\end{Highlighting}
\end{Shaded}

\includegraphics{scf_analysis_files/figure-latex/unnamed-chunk-17-1.pdf}

\hypertarget{insights-from-graph-3}{%
\subsubsection{Insights from Graph 3}\label{insights-from-graph-3}}

There are three instances of investors investing in accordance with
their risk aversion.

\begin{enumerate}
\def\labelenumi{\arabic{enumi}.}
\item
  The share of men who are invested in stocks increases with increased
  risk tolerance: 68.71\% of risk averse men own stocks, 72.88\% of risk
  neutral men own stocks, and 81.42\% of risk tolerant men own stocks.
  This makes sense because stocks are the riskiest of the four asssets.
\item
  The share of women who are invested in stock funds increases as women
  become risk tolerant: 47.33\% of risk averse women own stock funds,
  51.08\% of risk neutral women own stock funds, and 65.67\% of risk
  tolerant women own stock funds. This also makes sense because stock
  funds are the second riskiest of the four assets.
\item
  The share of risk averse women who own treasury bond funds is nearly
  twice that of the risk neutral and risk tolerant women who own
  treasury bond funds. This makes sense because treasury bond funds are
  the least risky of the four assets.
\end{enumerate}

Aside from these instances, there seems to be very little variation in
the percentage of each asset owned across the levels of risk aversion.
Thus, risk aversion may not have an effect on which assets an investor
chooses to invest in.

\hypertarget{portfolio-makeup-by-sex-and-risk-aversion}{%
\subsection{Portfolio Makeup, by Sex and Risk
Aversion}\label{portfolio-makeup-by-sex-and-risk-aversion}}

\hypertarget{table-4-percent_asset_sex_risk}{%
\subsubsection{Table 4:
percent\_asset\_sex\_risk}\label{table-4-percent_asset_sex_risk}}

Now, I am interested to analyze not just if the investor owns the
particular asset, but if they do, how much of their portfolio does the
asset makeup. Is the majority of their portfolio in one asset? Or are
they diversified with several types of assets?

I create a pivot table, ``percent\_asset\_sex\_risk,'' to observe this.

\begin{Shaded}
\begin{Highlighting}[]
\NormalTok{percent_asset_sex_risk <-}\StringTok{ }\NormalTok{scf_data_cleaned }\OperatorTok
\StringTok{  }\KeywordTok{select}\NormalTok{(}\DataTypeTok{Stocks=}\NormalTok{Percent_Stocks, }\DataTypeTok{Stock_Funds=}\NormalTok{Percent_Stock_Funds, }\DataTypeTok{Combination_Funds=}\NormalTok{Percent_Combination_Funds, }\DataTypeTok{Treasury_Bond_Funds=}\NormalTok{Percent_Treasury_Bond_Funds, Sex, Risk_Aversion) }\OperatorTok
\StringTok{  }\KeywordTok{pivot_longer}\NormalTok{(}\OperatorTok{-}\KeywordTok{c}\NormalTok{(Sex, Risk_Aversion), }\DataTypeTok{names_to=}\StringTok{'Asset'}\NormalTok{, }\DataTypeTok{values_to=}\StringTok{'Percentage_of_Portfolio'}\NormalTok{) }\OperatorTok
\StringTok{  }\KeywordTok{summarise}\NormalTok{(Sex, Asset, Risk_Aversion, Percentage_of_Portfolio)}
\end{Highlighting}
\end{Shaded}

I make the Asset variable a factor and order the assets from riskiest to
least risk.

\begin{Shaded}
\begin{Highlighting}[]
\NormalTok{percent_asset_sex_risk}\OperatorTok{$}\NormalTok{Asset <-}\StringTok{ }\KeywordTok{factor}\NormalTok{(percent_asset_sex_risk}\OperatorTok{$}\NormalTok{Asset, }\DataTypeTok{levels =} \KeywordTok{c}\NormalTok{(}\StringTok{"Stocks"}\NormalTok{, }\StringTok{"Stock_Funds"}\NormalTok{, }\StringTok{"Combination_Funds"}\NormalTok{, }\StringTok{"Treasury_Bond_Funds"}\NormalTok{, }\DataTypeTok{ordered =} \OtherTok{TRUE}\NormalTok{))}
\end{Highlighting}
\end{Shaded}

I filter ``percent\_asset\_sex\_risk,'' and create
``percent\_asset\_sex\_risk\_filtered,'' so that it only shows data for
investors who own each asset, which effectively is if the percentage of
their portfolio in that asset is greater than zero.

\begin{Shaded}
\begin{Highlighting}[]
\NormalTok{percent_asset_sex_risk_filtered <-}\StringTok{ }\NormalTok{percent_asset_sex_risk }\OperatorTok
\StringTok{  }\KeywordTok{filter}\NormalTok{(Percentage_of_Portfolio }\OperatorTok{>}\StringTok{ }\DecValTok{0}\NormalTok{)}
\end{Highlighting}
\end{Shaded}

\hypertarget{graph-4}{%
\subsubsection{Graph 4}\label{graph-4}}

I create boxplots to visualize the distribution of portfolio makeup
across the investors and grouped by sex and risk aversion.

\begin{Shaded}
\begin{Highlighting}[]
\KeywordTok{ggplot}\NormalTok{(percent_asset_sex_risk_filtered, }\KeywordTok{aes}\NormalTok{(}\DataTypeTok{x =}\NormalTok{ Sex, }\DataTypeTok{y =}\NormalTok{ Percentage_of_Portfolio, }\DataTypeTok{col =}\NormalTok{ Risk_Aversion))}\OperatorTok{+}
\StringTok{  }\KeywordTok{geom_boxplot}\NormalTok{()}\OperatorTok{+}\StringTok{ }
\StringTok{  }\KeywordTok{facet_wrap}\NormalTok{(}\OperatorTok{~}\NormalTok{Asset, }\DataTypeTok{ncol =} \DecValTok{4}\NormalTok{)}\OperatorTok{+}\StringTok{ }
\StringTok{  }\KeywordTok{labs}\NormalTok{(}\DataTypeTok{x =} \StringTok{"Sex"}\NormalTok{, }\DataTypeTok{y =} \StringTok{"Percentage of Investors' Portfolios"}\NormalTok{, }\DataTypeTok{title =} \StringTok{"Graph 4: Boxplots of the Percentage of Respondents' Portfolios in Each Asset, }\CharTok{\textbackslash{}n}\StringTok{ Grouped by Sex and Colored by Risk Aversion"}\NormalTok{)}\OperatorTok{+}
\StringTok{  }\KeywordTok{theme_classic}\NormalTok{()}\OperatorTok{+}
\StringTok{  }\KeywordTok{theme}\NormalTok{(}\DataTypeTok{legend.position =} \StringTok{"top"}\NormalTok{)}\OperatorTok{+}
\StringTok{  }\KeywordTok{scale_color_manual}\NormalTok{(}\DataTypeTok{values =} \KeywordTok{wes_palette}\NormalTok{(}\StringTok{"GrandBudapest1"}\NormalTok{, }\DataTypeTok{n =} \DecValTok{3}\NormalTok{))}
\end{Highlighting}
\end{Shaded}

\includegraphics{scf_analysis_files/figure-latex/unnamed-chunk-21-1.pdf}

\hypertarget{insights-from-graph-4}{%
\subsubsection{Insights from Graph 4}\label{insights-from-graph-4}}

These results are pretty interesting!

It seems that for most investors who own stocks, the majority of their
portfolio is in stocks. For both men and women, the 50th and 75th
percentiles are at 100\% for all levels of risk aversion.

The results for stock funds are particularly interesting because we see
distinct differences between male and female investors. The 50th and
75th percentiles are still at 100\%, but for male investors, the 25th
and 50th percentiles decrease as male investors become more comfortable
with risk.

For Combination Funds and Treasury Bond Funds, the 25th, 50th, and 75th
percentiles decrease as both male and female investors become more
comfortable with risk. Combination and Treasury Bond Funds are the least
risky. Thus, it seems that investors who are risk averse own a greater
share of their portfolio in the least risky assets. Logically, this
makes a lot of sense.

\hypertarget{tests-for-statistical-significance}{%
\section{Tests for Statistical
Significance}\label{tests-for-statistical-significance}}

I use the Cochran-Mantel-Haenszel test to analyze if it is statistically
significant that one sex is more likely than the other to invest than
the other while controlling for Risk Aversion. The variables I will be
using are Sex, Asset, Risk Aversion, and Owns -- whether or not the
investor owns that particular investment.

The Cochran-Mantel-Haenszel test is useful because it is used for
categorical variables, can analyze up to three variables at a time (a
chi-squared test can only analyze two at a time), and it controls for
how there are six times as many male investors as female investors.

First, I will run this test on all of the asset classes together, and
then each asset class individually. I run the extra tests because I am
curious to see if the results change for each asset class.

\hypertarget{all-asset-classes}{%
\subsection{All Asset Classes}\label{all-asset-classes}}

\hypertarget{create-table}{%
\subsubsection{Create Table}\label{create-table}}

I start by creating a data table, ``dt,'' to organize the data which I
will use throughout the process of checking for statistical
significance. This includes the count and percentage for each sex, level
of risk aversion, and asset owned.

For example: How many women are risk averse and owned stocks? What
percentage of men are risk neutral and own combination funds?

\begin{Shaded}
\begin{Highlighting}[]
\NormalTok{dt <-}\StringTok{ }\NormalTok{scf_data_cleaned }\OperatorTok\StringTok{ }\KeywordTok{select}\NormalTok{(Sex, }\DataTypeTok{Treasury_Bond_Funds=}\NormalTok{Owns_Treasury_Bond_Funds, }\DataTypeTok{Stocks=}\NormalTok{Owns_Stocks, }\DataTypeTok{Stock_Funds=}\NormalTok{Owns_Stock_Funds, }\DataTypeTok{Combination_Funds=}\NormalTok{Owns_Combination_Funds, Risk_Aversion) }\OperatorTok\StringTok{ }\KeywordTok{pivot_longer}\NormalTok{(}\OperatorTok{-}\KeywordTok{c}\NormalTok{(Sex, Risk_Aversion), }\DataTypeTok{names_to=}\StringTok{'Asset'}\NormalTok{, }\DataTypeTok{values_to=}\StringTok{'Owns'}\NormalTok{) }\OperatorTok
\StringTok{  }\KeywordTok{group_by}\NormalTok{(Sex, Asset, Risk_Aversion, Owns) }\OperatorTok
\StringTok{  }\KeywordTok{summarise}\NormalTok{(}\DataTypeTok{Number_of_Responses=}\KeywordTok{n}\NormalTok{()) }\OperatorTok
\StringTok{  }\KeywordTok{group_by}\NormalTok{(Sex, Asset, Risk_Aversion) }\OperatorTok
\StringTok{  }\KeywordTok{mutate}\NormalTok{(}\DataTypeTok{Percentage=}\NormalTok{Number_of_Responses}\OperatorTok{/}\KeywordTok{sum}\NormalTok{(Number_of_Responses)}\OperatorTok{*}\DecValTok{100}\NormalTok{)}
\end{Highlighting}
\end{Shaded}

\begin{verbatim}
## `summarise()` regrouping output by 'Sex', 'Asset', 'Risk_Aversion' (override with `.groups` argument)
\end{verbatim}

Here, I combine the Asset and Risk Aversion variables into one column
because the Cochran-Mantel-Haenszel test can only analyze three
categorical variables at a time. Thus, "Asset\_\_Risk\_Aversion,"
collapses the two variables into one. Also, I condense the table by not
selecting ``Percentage.''

\begin{Shaded}
\begin{Highlighting}[]
\NormalTok{dt_all <-}\StringTok{ }\KeywordTok{xtabs}\NormalTok{(Number_of_Responses }\OperatorTok{~}\StringTok{ }\NormalTok{Asset__Risk_Aversion }\OperatorTok{+}\NormalTok{Sex }\OperatorTok{+}\StringTok{ }\NormalTok{Owns, }\DataTypeTok{data =}\NormalTok{ dt }\OperatorTok\StringTok{ }\KeywordTok{mutate}\NormalTok{(}\DataTypeTok{Asset__Risk_Aversion =} \KeywordTok{paste0}\NormalTok{(Asset,}\StringTok{'_'}\NormalTok{, Risk_Aversion)))}
\end{Highlighting}
\end{Shaded}

This is what the data looks like.

\begin{Shaded}
\begin{Highlighting}[]
\KeywordTok{ftable}\NormalTok{(dt_all)}
\end{Highlighting}
\end{Shaded}

\begin{verbatim}
##                                          Owns Does Not Own Owns
## Asset__Risk_Aversion              Sex                          
## Combination_Funds_Risk Averse     Female               471   34
##                                   Male                1490  105
## Combination_Funds_Risk Neutral    Female               826   53
##                                   Male                3449  341
## Combination_Funds_Risk Tolerant   Female               218   15
##                                   Male                1977  177
## Stock_Funds_Risk Averse           Female               266  239
##                                   Male                 823  772
## Stock_Funds_Risk Neutral          Female               430  449
##                                   Male                1522 2268
## Stock_Funds_Risk Tolerant         Female                80  153
##                                   Male                 947 1207
## Stocks_Risk Averse                Female               164  341
##                                   Male                 499 1096
## Stocks_Risk Neutral               Female               258  621
##                                   Male                1028 2762
## Stocks_Risk Tolerant              Female                90  143
##                                   Male                 400 1754
## Treasury_Bond_Funds_Risk Averse   Female               459   46
##                                   Male                1457  138
## Treasury_Bond_Funds_Risk Neutral  Female               840   39
##                                   Male                3431  359
## Treasury_Bond_Funds_Risk Tolerant Female               220   13
##                                   Male                1972  182
\end{verbatim}

\hypertarget{mantelhaen-test}{%
\subsubsection{Mantelhaen Test}\label{mantelhaen-test}}

\begin{Shaded}
\begin{Highlighting}[]
\KeywordTok{mantelhaen.test}\NormalTok{(dt_all)}
\end{Highlighting}
\end{Shaded}

\begin{verbatim}
## 
##  Cochran-Mantel-Haenszel test
## 
## data:  dt_all
## Cochran-Mantel-Haenszel M^2 = 667.77, df = 11, p-value < 2.2e-16
\end{verbatim}

\hypertarget{findings}{%
\subsubsection{Findings}\label{findings}}

It is statistically significant than men are more likely to invest than
women.

\hypertarget{stocks}{%
\subsection{Stocks}\label{stocks}}

Next, I analyze just for stock ownership.

\hypertarget{create-tabel}{%
\subsubsection{Create Tabel}\label{create-tabel}}

I filter the original data table, ``dt'' for only stocks.

\begin{Shaded}
\begin{Highlighting}[]
\NormalTok{dt_s <-}\StringTok{ }\NormalTok{dt }\OperatorTok
\StringTok{  }\KeywordTok{filter}\NormalTok{(Asset}\OperatorTok{==}\StringTok{"Stocks"}\NormalTok{)}
\end{Highlighting}
\end{Shaded}

I condense the data because I no longer need the ``Percentage''
variable.

\begin{Shaded}
\begin{Highlighting}[]
\NormalTok{dt_stocks <-}\StringTok{ }\KeywordTok{xtabs}\NormalTok{(Number_of_Responses }\OperatorTok{~}\StringTok{ }\NormalTok{Risk_Aversion}\OperatorTok{+}\NormalTok{Sex}\OperatorTok{+}\NormalTok{Owns, }\DataTypeTok{data =}\NormalTok{ dt_s)}
\end{Highlighting}
\end{Shaded}

This is what the data looks like.

\begin{Shaded}
\begin{Highlighting}[]
\KeywordTok{ftable}\NormalTok{(dt_stocks)}
\end{Highlighting}
\end{Shaded}

\begin{verbatim}
##                      Owns Does Not Own Owns
## Risk_Aversion Sex                          
## Risk Averse   Female               164  341
##               Male                 499 1096
## Risk Neutral  Female               258  621
##               Male                1028 2762
## Risk Tolerant Female                90  143
##               Male                 400 1754
\end{verbatim}

\hypertarget{mantelhaen-test-1}{%
\subsubsection{Mantelhaen Test}\label{mantelhaen-test-1}}

\begin{Shaded}
\begin{Highlighting}[]
\KeywordTok{mantelhaen.test}\NormalTok{(dt_stocks)}
\end{Highlighting}
\end{Shaded}

\begin{verbatim}
## 
##  Cochran-Mantel-Haenszel test
## 
## data:  dt_stocks
## Cochran-Mantel-Haenszel M^2 = 156.66, df = 2, p-value < 2.2e-16
\end{verbatim}

\hypertarget{findings-1}{%
\subsubsection{Findings}\label{findings-1}}

It is a statistically significant finding that men are more likely to
invest in stocks than women.

\hypertarget{stock-funds}{%
\subsection{Stock Funds}\label{stock-funds}}

Next, I analyze for stock funds.

\hypertarget{create-table-1}{%
\subsubsection{Create Table}\label{create-table-1}}

I filter the original data table, ``dt'' for only stock funds.

\begin{Shaded}
\begin{Highlighting}[]
\NormalTok{dt_sf <-}\StringTok{ }\NormalTok{dt }\OperatorTok
\StringTok{  }\KeywordTok{filter}\NormalTok{(Asset}\OperatorTok{==}\StringTok{"Stock_Funds"}\NormalTok{)}
\end{Highlighting}
\end{Shaded}

I condense the data because I no longer need the ``Percentage''
variable.

\begin{Shaded}
\begin{Highlighting}[]
\NormalTok{dt_stock_funds <-}\StringTok{ }\KeywordTok{xtabs}\NormalTok{(Number_of_Responses }\OperatorTok{~}\StringTok{ }\NormalTok{Risk_Aversion}\OperatorTok{+}\NormalTok{Sex}\OperatorTok{+}\NormalTok{Owns, }\DataTypeTok{data =}\NormalTok{ dt_sf)}
\end{Highlighting}
\end{Shaded}

This is what the data looks like.

\begin{Shaded}
\begin{Highlighting}[]
\KeywordTok{ftable}\NormalTok{(dt_stock_funds)}
\end{Highlighting}
\end{Shaded}

\begin{verbatim}
##                      Owns Does Not Own Owns
## Risk_Aversion Sex                          
## Risk Averse   Female               266  239
##               Male                 823  772
## Risk Neutral  Female               430  449
##               Male                1522 2268
## Risk Tolerant Female                80  153
##               Male                 947 1207
\end{verbatim}

\hypertarget{mantelhaen-test-2}{%
\subsubsection{Mantelhaen test}\label{mantelhaen-test-2}}

\begin{Shaded}
\begin{Highlighting}[]
\KeywordTok{mantelhaen.test}\NormalTok{(dt_stock_funds)}
\end{Highlighting}
\end{Shaded}

\begin{verbatim}
## 
##  Cochran-Mantel-Haenszel test
## 
## data:  dt_stock_funds
## Cochran-Mantel-Haenszel M^2 = 162.07, df = 2, p-value < 2.2e-16
\end{verbatim}

\hypertarget{findings-2}{%
\subsubsection{Findings}\label{findings-2}}

It is a statistically significant finding that men are more likely to
invest in stock funds than women.

\hypertarget{combination-funds}{%
\subsection{Combination Funds}\label{combination-funds}}

\hypertarget{create-table-2}{%
\subsubsection{Create Table}\label{create-table-2}}

I filter the original data table, ``dt'' for only combination funds.

\begin{Shaded}
\begin{Highlighting}[]
\NormalTok{dt_cf <-}\StringTok{ }\NormalTok{dt }\OperatorTok
\StringTok{  }\KeywordTok{filter}\NormalTok{(Asset}\OperatorTok{==}\StringTok{"Combination_Funds"}\NormalTok{)}
\end{Highlighting}
\end{Shaded}

I condense the data because I no longer need the ``Percentage''
variable.

\begin{Shaded}
\begin{Highlighting}[]
\NormalTok{dt_combination_funds <-}\StringTok{ }\KeywordTok{xtabs}\NormalTok{(Number_of_Responses }\OperatorTok{~}\StringTok{ }\NormalTok{Risk_Aversion}\OperatorTok{+}\NormalTok{Sex}\OperatorTok{+}\NormalTok{Owns, }\DataTypeTok{data =}\NormalTok{ dt_cf)}
\end{Highlighting}
\end{Shaded}

This is what the data looks like.

\begin{Shaded}
\begin{Highlighting}[]
\KeywordTok{ftable}\NormalTok{(dt_combination_funds)}
\end{Highlighting}
\end{Shaded}

\begin{verbatim}
##                      Owns Does Not Own Owns
## Risk_Aversion Sex                          
## Risk Averse   Female               471   34
##               Male                1490  105
## Risk Neutral  Female               826   53
##               Male                3449  341
## Risk Tolerant Female               218   15
##               Male                1977  177
\end{verbatim}

\#\#\#Mantelhaen test

\begin{Shaded}
\begin{Highlighting}[]
\KeywordTok{mantelhaen.test}\NormalTok{(dt_combination_funds)}
\end{Highlighting}
\end{Shaded}

\begin{verbatim}
## 
##  Cochran-Mantel-Haenszel test
## 
## data:  dt_combination_funds
## Cochran-Mantel-Haenszel M^2 = 164.81, df = 2, p-value < 2.2e-16
\end{verbatim}

\hypertarget{findings-3}{%
\subsubsection{Findings}\label{findings-3}}

It is a statistically significant finding that men are more likely to
invest in combination funds than women.

\hypertarget{treasury-bond-funds}{%
\subsection{Treasury Bond Funds}\label{treasury-bond-funds}}

\hypertarget{create-table-3}{%
\subsubsection{Create Table}\label{create-table-3}}

I filter the original data table, ``dt'' for only treasury bond funds.

\begin{Shaded}
\begin{Highlighting}[]
\NormalTok{dt_tbf <-}\StringTok{ }\NormalTok{dt }\OperatorTok
\StringTok{  }\KeywordTok{filter}\NormalTok{(Asset}\OperatorTok{==}\StringTok{"Treasury_Bond_Funds"}\NormalTok{)}
\end{Highlighting}
\end{Shaded}

I condense the data because I no longer need the ``Percentage''
variable.

\begin{Shaded}
\begin{Highlighting}[]
\NormalTok{dt_treasury_bond_funds <-}\StringTok{ }\KeywordTok{xtabs}\NormalTok{(Number_of_Responses }\OperatorTok{~}\StringTok{ }\NormalTok{Risk_Aversion}\OperatorTok{+}\NormalTok{Sex}\OperatorTok{+}\NormalTok{Owns, }\DataTypeTok{data =}\NormalTok{ dt_tbf)}
\end{Highlighting}
\end{Shaded}

This is what the data looks like.

\begin{Shaded}
\begin{Highlighting}[]
\KeywordTok{ftable}\NormalTok{(dt_treasury_bond_funds)}
\end{Highlighting}
\end{Shaded}

\begin{verbatim}
##                      Owns Does Not Own Owns
## Risk_Aversion Sex                          
## Risk Averse   Female               459   46
##               Male                1457  138
## Risk Neutral  Female               840   39
##               Male                3431  359
## Risk Tolerant Female               220   13
##               Male                1972  182
\end{verbatim}

\#\#\#Mantelhaen Test

\begin{Shaded}
\begin{Highlighting}[]
\KeywordTok{mantelhaen.test}\NormalTok{(dt_treasury_bond_funds)}
\end{Highlighting}
\end{Shaded}

\begin{verbatim}
## 
##  Cochran-Mantel-Haenszel test
## 
## data:  dt_treasury_bond_funds
## Cochran-Mantel-Haenszel M^2 = 166.66, df = 2, p-value < 2.2e-16
\end{verbatim}

\hypertarget{findings-4}{%
\subsubsection{Findings}\label{findings-4}}

It is a statistically significant finding that men are more likely to
invest in treasury bond funds than women.

\hypertarget{conclusion}{%
\section{Conclusion}\label{conclusion}}

Through the Cochran-Mantel-Haenszel tests, we can see that in this data
set, men are more likely to invest than women overall and for each asset
class. The visualizations show that there is a greater share of women
who are risk averse and risk neutral than men, and thus there are a
greater share of men who are risk tolerant. There are a few examples of
risk tolerant investors having a bigger portion of their portfolios in
risky assets like stocks and stock funds, as well as risk averse
investors having a bigger portion of their portfolios in risk averse
assets like treasury bond funds. However, for the most part,
self-reported risk aversion doesn't seem to have a big impact on
portfolio makeup.

There is an important caveat to this analysis, however. Though stocks
generally are the riskiest asset, there is a wide variety of risk levels
for different types of stocks. For example, an investor could own a
share of a Blue Chip stock that is not volatile and is known for having
consistent and reliable growth -- which would make it not very risky.
Another investor could own a stock fund where the underlying stocks in
the fund are very volatile -- making it very risky. So in this example,
the stock fund is actually riskier than the stock. Unfortunately, there
is no way to control for this using this data because the Survey of
Consumer Finances does not ask specific questions about the risk level
of the assets they own.

\end{document}
